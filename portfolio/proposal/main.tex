\documentclass[10pt]{article}

\usepackage{graphicx} % Required for inserting images
\usepackage{setspace}
\setstretch{0.85} % slightly tighter line spacing

\title{KCR - predlog projekta}
\author{Maks P.V., Mark L., Luka P., Domen O., Tilen V.}
\date{October 2025}

\begin{document}

\maketitle

\section{Uvod}
Glavna tema projekte naloge je posodobitev izgleda prve strani znane in priljubljene slovenske spletne strani \textit{hribi.net}. Gre za spletni portal, ki združuje vse ključne informacije za ljubitelje pohodništva po slovenskih in bližnjih (avstrijskih, italijanskih) gorah. Prva stran trenutno vključuje novice, forume, zbirko izletov, pogovorne niti, oglase za opremo, spletne kamere, objave poti na zemljevidu in številne druge vsebine. Kljub bogati ponudbi informacij pa je oblikovno in tehnično zastarela.
Cilj projekta je prenova uporabniškega vmesnika s pomočjo sodobnih metodologij razvoja spletnih rešitev. To vključuje raziskavo uporabniških potreb, analizo podobnih rešitev, izvedbo anket ter uporabo sodobnih pristopov oblikovanja za izboljšano uporabniško izkušnjo.

\section{Cilji projekta in uporabljena metodologija}
\begin{enumerate}
    \item Modernizacija spletnega vmesnika;
    \item Anketiranje uporabnikov za izboljšanje obstoječih funkcij in dodajanje novih za izboljšanje uporabniške izkušnje;
    \item Izboljšati pregled nad vsemi informacijami, ki jih stran ponuja.
\end{enumerate}

\section{Načrt izvedbe}
Projekta za izboljšavo spletnega vmesnika se bomo lotili po naslednjih korakih:
\begin{enumerate}
    \item \textbf{Pregled področja}: analiza uporabniških vmesnikov podobnih domačih in tujih spletnih strani ter določitev njihovih prednosti in slabosti;
    \item \textbf{Anketiranje uporabnikov}: priprava in izvedba ankete med uporabniki portala \textit{hribi.net} z namenom zbiranja mnenj in prepoznavanja težav trenutnega vmesnika;
    \item \textbf{Prototipiranje}: izdelava zaslonskih mask in predlogov novega spletnega vmesnika;
    \item \textbf{Evalvacija prototipov}: ponovna presoja in izboljšava izdelanih prototipov na podlagi pridobljenih odzivov;
    \item \textbf{Implementacija}: razvoj in implementacija posodobljenega spletnega vmesnika.
\end{enumerate}

\end{document}

